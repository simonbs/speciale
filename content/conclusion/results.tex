\section{Project Results}
\label{sec:results}

In this section, we will conclude on the results from investigating the problem statement presented in \Cref{sec:problem-statement}:

\begin{framed}
\noindent How can we design and implement a system that utilizes contextual information for controlling a smart home using a wearable in a gesture driven solution?
\end{framed}

We have presented an approach for recognizing gestures and once a gesture is recognized beginning to recognize the context and trigger an appropriate action. The system presented has been designed and implemented on an Android Wear smartwatch and a Raspberry Pi, allowing users to control a music centre and smart bulbs.

We intended to model a generic engine for recognizing context based on various sources for contextual information. Our concrete implementation of the system utilizes BLE for positioning the user and a gesture recognizer derived from 1\textcent~\cite{herold20121} and \$3~\cite{threedollar}. In practice the context engine suggests different actions for the same gesture depending on the position of the user in his smart home.

As described in \Cref{sec:evaluation:user-tests}, the system triggers the correct action 44\% of the time. According to the requirement specification presented in \Cref{sec:requirements-specification}, an accuracy of at least 80\% was desired and as such, the system is not satisfyingly accurate.

We presented a concrete design and implementation for a system utilizing contextual information to control a smart home. As the accuracy of the system is unsatisfying, better designs of such a system may exist. We have investigated an alternative design of the Bayesian network and found that it was not more accurate.
We also presented an influence diagram to potentially replace the Bayesian network. The diagram proved to perform better when tested with data from a single participant.
Further work on this project should investigate the possibility of using influence diagrams as the model for the context engine.

%%% Local Variables:
%%% mode: latex
%%% TeX-master: "../../master"
%%% End:
