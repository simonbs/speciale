\section{Alternative Models}
\label{sec:evaluation:alternative-models}

As the users tests described in \Cref{sec:evaluation:user-tests} showed that the system had an accuracy of 44\%. This section describes some of the alternative models that were considered and whether or not they addressed the issues raised in \Cref{sec:evaluation:user-tests-conclusion}.

\subsection{Bayesian Network With a Single Action Node}

The Bayesian network used on the context engine consists of an Action node and its parent nodes are Gesture\_Action and Room\_Action.
These two parent nodes were intended to make the network more modular to support additional context information later but in this model the probability of each action is calculated multiple times based on different prior probabilities.
Hence we considered an alternate model in which the Gesture and Room nodes were direct parents of the Action node.

\begin{figure}[h]
\centering
\includegraphics[width=0.40\textwidth]{images/DirectParentBaysianNetwork}
\caption{Bayesian network without the intermediate \emph{gesture\_action} and \emph{room\_action} nodes.}
\label{fig:evaluation:alternative-models:direct-parents}
\end{figure}

This alternate network was tested with the belief values of the Gesture and Room nodes of participant 7 and a new success rate of 36\% was computed.
Compared to the previous success rate of 41\% for the participant, this model did not prove to be an improvement over the existing one and was discarded.

\subsection{Influence Diagram}

Influence diagrams can be regarded as Bayesian networks extended with decision variables and utility functions~\cite{kjaerulff2008bayesian}. \Cref{fig:evaluation:alternative-models:influence-diagram} shows an influence diagram modelling a context engine. The model does not correspond exactly with the model previously presented in this report but is an alternative model.

As for the graphical representation of an influence diagram, the square nodes represent \emph{decision nodes}, \ie something we must decide to do or not to do. Diamond nodes represent \emph{utility nodes} describing a \emph{utility function} and oval nodes represent \emph{uncertainty nodes}. In an influence diagram we are generally interested in taking the decision that results in the highest utility, therefore utility can be regarded a measurement of the quality of a decision.

In the model presented in \Cref{fig:evaluation:alternative-models:influence-diagram}, the utility is a function of the gesture, room and action. We assign a high utility to combinations of gesture and room that are part of a gesture configuration. 

Influence diagrams provide a natural way of including the system state, as actions which it does not make sense to trigger given the current state of the system, can be assigned a very low utility.

\begin{figure}[!h]
\centering
\includegraphics[width=0.50\textwidth]{images/influence-diagram}
\caption{Example model of the context engine using an influence diagram.}
\label{fig:evaluation:alternative-models:influence-diagram}
\end{figure}

Due to inference in the probabilistic network, actions are assigned a utility when soft or hard evidence on the gesture and room nodes is available. The utility of an action is shown with dark green bars below the name of the action in \Cref{fig:evaluation:alternative-models:hugin-influence-diagram}. For example, the utility of the \emph{Shelves\_Lamp\_Toggle} is 6682, making it the action with the highest utility and thus the action that should be triggered\footnote{More information on influence diagrams in Hugin is available at \url{http://www.hugin.com/technology/getting-started/ids}}.

The scenario discussed in \Cref{sec:evaluation:user-tests} in which the belief values of a gesture was divided when the gesture was associated with multiple different actions is addressed in \Cref{fig:evaluation:alternative-models:hugin-influence-diagram} where the correct action has a greater utility and can be triggered with greater confidence.

When testing the Bayesian network with a subset of the recorded data from the user tests, we found that when using the influence diagram, we were able to trigger the correct action more often. During the user test a correct action was triggered 36\% of the time for participant 7. Using the exact same beliefs on gesture and room nodes in the influence diagram, the correct action is triggered 50\% of the time.

% Note that during the tests with the Bayesian network we considered the outcome of performing an action acceptable if a list of suggested actions were shown and the intended action was in the list. This was not taken into account when testing the influence diagram with the data from participant 7 and as such the accuracy of the influence diagram may increase if a threshold for the utility of accepted actions is introduced.

\begin{figure}[h]
\centering
\includegraphics[width=\textwidth]{images/hugin-influence-diagram}
\caption{Screenshot of example influence diagram in Hugin. The screenshot shows that the influence diagram solves the problem of a single gesture being bound to multiple actions, can result in an incorrect action being triggered because the belief is reduced as shown in \Cref{table:participant-2-first-run} and discussed in \Cref{sec:evaluation:user-tests}.}
\label{fig:evaluation:alternative-models:hugin-influence-diagram}
\end{figure}

\subsection{Calculation of Gesture Beliefs}

In \Cref{sec:design:bayesian-network:gesture-node-evidence} the computation of beliefs on the Gesture node in the Bayesian network is specified. We only consider gesture templates with a score of 70 or below and then compute the average score. We found that this approach may cause issues, because a single gesture template may have a very low score but all other gesture templates with the same name have very high scores, thus the single gesture template is an outlier potentially causing us to consider an incorrect gesture recognized.

Instead we propose computing the average before filtering the gesture templates. Naturally, the average of each gesture will be larger as we include templates with a higher score in the computation and as such the threshold for accepted gesture should be higher.

Using the scores of the gestures performed by participant 7 in the user test, we computed how many actions would be correctly triggered when computing the average of the scores first and then filter them based on a threshold. We chose a threshold of 100. We found that 45\% of the actions were correctly triggered when we adjusted the computations of the gesture beliefs. This is in contrast to 41\% before the adjustments were made. While the adjustments are not a big impact, this can be considered an indication that the computations of beliefs in the Gesture node should have been performed differently.

We also found that when changing the way we compute beliefs for the Gesture node, we are generally able to put a greater belief on the correct action. This is examplified in \Cref{sec:evaluation:alternative-models:improved-action-beliefs}. The beliefs are taken from a scenario where the user desires to trigger the \emph{Spotify\_PlayPause} action. Note that in the beliefs before the improvements to the computations, \emph{Spotify\_Next} has the greatest belief where as after the improvements are made, the \emph{Spotify\_PlayPause} has the highest belief.
The average score used when computing the improved beliefs are shown in \Cref{sec:evaluation:alternative-models:improved-gesture-beliefs}. Note that only the Circle and V gestures have an average score below 100 and thus only those two gestures are assigned a belief greater than zero. Before improving the computations, all four gestures were assigned a belief greater than zero.

\begin{table}[]
\centering
\caption{Example beliefs of the Gesture node for participant 7 when attempting to trigger an action. The beliefs are shown before the computations were improved and after.}
\label{sec:evaluation:alternative-models:improved-action-beliefs}
\begin{tabular}{lll}
\textbf{Action}         & \textbf{Belief before} & \textbf{Belief after} \\
Spotify\_PlayPause      & 28.39                                              & 42.24                                             \\
Spotify\_Next           & 30.06                                              & 16.67                                             \\
Shelves\_Lamp\_Toggle   & 22.87                                              & 28.88                                             \\
Architect\_Lamp\_Toggle & 6.20                                               & 12.21                                             \\
TV\_Lamp\_Toggle        & 12.48                                              & 0                                                 
\end{tabular}
\end{table}

\begin{table}[]
\centering
\caption{Average scores of gestures used when computing the beliefs for the ``Belief after'' column shown in \Cref{sec:evaluation:alternative-models:improved-action-beliefs}.}
\label{sec:evaluation:alternative-models:improved-gesture-beliefs}
\begin{tabular}{ll}
\textbf{Gesture}    & \textbf{Score} \\
Circle              & 89.80          \\
Swipe Left to Right & 116.03         \\
V                   & 94.03          \\
Zorro               & 110.54         
\end{tabular}
\end{table}

%%% Local Variables:
%%% mode: latex
%%% TeX-master: "../../master"
%%% End:
