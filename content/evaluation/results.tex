\section{User Tests}
\label{sec:evaluation:user-tests}

This section will describe the experimental procedures carried out during the project to determine how well it performed when used by
other people and how well it performed comparing to the requirements specified in \cref{sec:requirements-specification}.

\subsection{Setup}
\label{sec:evaluation:user-tests-setup}

The setup of our user test consisted of a Macbook Pro running OpenHAB and Spotify, two Philips Hue lamps and two Estimote beacons.
Seven people were asked to train four unique gestures, creating ten templates per gesture resulting in a total of 40 gesture templates
per person.
The gestures used were:

\begin{itemize}
  \item Circle
  \item Swipe left to right
  \item V
  \item Zorro
\end{itemize}

The setup was limited to a subset of the smart devices and gestures presented in the scenario \cref{sec:analysis:scenarios}, because we did not have the necessary hardware and space available to perform a test of the entire scenario as well as to keep the invested time for each participant to a minimum, so it is more likely that they schedule time to participate in a test.

The gesture templates stored in the database were removed before each test so each participant was only using his own templates and not those created by others.

The participants were instructed how to perform the gestures but were allowed to scale them to their personal preference, eg.
either create large circles or small circles.

Once the gesture templates had been created the participants were asked to complete seven tasks, each of which required them to perform a given gesture five times.
Tests took place in a single room but to simulate the user moving between two different rooms, only one Estimote beacon would be turned on at a time.
A list of the gestures and locations is shown in in \cref{table:user-test-tasks}, along with the OpenHAB actions they were supposed to trigger.

\begin{table}[]
\centering
\begin{tabular}{|l|l|l|}
\hline
\textbf{Action}                  & \textbf{Gesture}             & \textbf{Location}                       \\ \hline
Shelves\_Lamp\_Toggle   & V                   & Home Office                    \\ \hline
Spotify\_PlayPause      & Circle              & Home Office                    \\ \hline
Spotify\_Next           & Swipe Left to Right & Home Office                    \\ \hline
Architect\_Lamp\_Toggle & V                   & Living Room                    \\ \hline
TV\_Lamp\_Toggle        & Zorro               & Living Room                    \\ \hline
Spotify\_Next           & Swipe Left to Right & Home Office (Virtual Position) \\ \hline
Shelves\_Lamp\_Toggle   & V                   & Home Office                    \\ \hline
\end{tabular}
\caption{The actions, gestures and locations that were used during the user tests.}
\label{table:user-test-tasks}
\end{table}

\subsection{Results}
\label{sec:evaluation:user-tests-results}

For each participant and task, the correctness rate of the actions triggered was calculated as the \emph{number of times the intended action was triggered} divided by the \emph{number of attempts} and can be found in \cref{fig:user-test-action-correctness}.
In the same manner, the correctness rate for gestures and locations have been plotted in \cref{fig:user-test-gesture-correctness,fig:user-test-location-correctness}.
For each participant, the average correctness rate for all actions, gestures, and locations can be found in \cref{table:user-task-averages}.
\Cref{fig:participant-average} shows the combined averages of all participents separated per task.
This shows that the locations is correctly identified in the majority of the cases with an average correctness rate of 0.827020202 which indicates that the positioning is reliable.
The average correctness rate of actions is below the 80\% requirement specified in \cref{sec:requirements-specification} with a combined average of 0.4361381674.
Triggering the correct action less than half of the time is not a satisfactory result and is thus something that needs to be looked into.

One source of the inaccuracy comes from the low average correctness rates of the gestures.
\cref{fig:participant-average} shows that the gesture correctness rates are generally higher than those for actions with an average of 0.5567550505.
However differences between the average gesture correctness rates of the participants are noticable.
The most noticable example is comparing \cref{fig:participant-1,fig:participant-2}.
Something to note here is that both of these participants, as well as participant 1 experienced some technical difficulties which prevented them from completing the task.
Participant 2 has an average correctness rate for gestures of \emph{0.925} and participant 6 has an average correctness rate of \emph{0.2}.
This is a significant difference and indicates that possibly something went wrong during the tests of participant 6, or the gesture recognizer is not capable of handling people with different capabilities of performing these gestures.

Another source of the inaccuracy of actions is the way we have modeled the context engine.
Loking at the corectness rates for \emph{Shelves_Lamp_Toggle} in \cref{fig:participant-1} shows that the correct gesture and location was always identified, yet the correct action was only triggered 40\% of the time.

In a single trial of participant 2 the scores in \cref{table:participant-2-first-run} were recored.
The gesture \emph{V} correctly received the highest belief in the \emph{Gesture} node but because \emph{V} is configured to be used with two actions, \emph{Shelves\_Lamp\_Toggle} as well as \emph{Architect\_Lamp\_Toggle}, the \emph{Gesture\_Action} node divides the belief amongst these.
The  gesture \emph{Circle} did not match as well as \emph{V} and only recieved approximately half of the belief value of \emph{V}.
However, since \emph{Circle} is only configured to trigger a single action, its belief value remains intact in the \emph{Gesture\_Action} node.
This results in the beliefs of \emph{Spotify\_PlayPause} and \emph{Shelves\_Lamp\_Toggle} being approximately equivalent even after the location beliefs have been applied.

As such it is more difficult to properly trigger actions where the gesture is configured to other actions as well, than triggering actions where the gesture is only configured for that one.

As such it is easier to trigger actions that only have a single gesture associated with them than actions that have multiple.

\begin{table}[]
\centering
\begin{tabular}{|l|l|}
\hline
\textbf{Intended Action}   & Shelves_Lamp_toggle \\ \hline
\textbf{Intended Gesture}  & V                   \\ \hline
\textbf{Intended Location} & Home_Office         \\ \hline
\textbf{Gesture}           & \textbf{Belief}     \\ \hline
Circle                     & 33.18003597         \\ \hline
Swipe Left to Right        & 0                   \\ \hline
V                          & 66.81996403         \\ \hline
Zorro                      & 0                   \\ \hline
\textbf{Gesture\_Action}   &                     \\ \hline
Spotify\_PlayPause         & 33.18003597         \\ \hline
Spotify\_Next              & 0                   \\ \hline
Shelves\_Lamp\_Toggle      & 33.40998201         \\ \hline
Architect\_Lamp\_Toggle    & 33.40998201         \\ \hline
TV\_Lamp\_Toggle           & 0                   \\ \hline
\textbf{Room}              &                     \\ \hline
Living Room                & 0                   \\ \hline
Home Office                & 100                 \\ \hline
\textbf{Room\_Action}      &                     \\ \hline
Spotify\_PlayPause         & 33.33333333         \\ \hline
Spotify\_Next              & 33.33333333         \\ \hline
Shelves\_Lamp\_Toggle      & 33.33333333         \\ \hline
Architect\_Lamp\_Toggle    & 0                   \\ \hline
TV\_Lamp\_Toggle           & 0                   \\ \hline
\textbf{Action}            &                     \\ \hline
Spotify\_PlayPause         & 33.25668465         \\ \hline
Spotify\_Next              & 16.66666667         \\ \hline
Shelves\_Lamp\_Toggle      & 33.37165767         \\ \hline
Architect\_Lamp\_Toggle    & 16.70499101         \\ \hline
TV\_Lamp\_Toggle           & 0                   \\ \hline
\end{tabular}
\caption{Belief values for the different nodes in the bayesian network of the context engine for a single trial of Participant 2.}
\label{table:participant-2-first-run}
\end{table}


\begin{figure}[]
\begin{longtable}{p{0.45\textwidth} p{0.45\textwidth}}
  \centering
  \begin{tikzpicture}
  \begin{axis}[
    ybar,
    bar width=2pt,
    xticklabels from table={data/ActionCorrectnessTransposed.csv}{Action},
    xtick=data,
    x tick label style={
    rotate=45,
    anchor=east,
    },
    width=0.40\textwidth,
    ymin = 0,
    ymax = 1,
    ylabel = correctness rate,
    legend style={
    at={(0,0)},
    anchor=west, at={(axis description cs:1,0.5)}}]
    \addplot table[x=Row, y=2] {data/GestureCorrectness.csv};
    \addplot table[x=Row, y=2] {data/ActionCorrectnessTransposed.csv};
    \addplot table[x=Row, y=2] {data/LocationCorrectness.csv};
    \legend{Gesture, Action, Location}
  \end{axis}
  \end{tikzpicture}
  \figcaption{Participant 2.}
  \label{fig:participant-1}
&
  \begin{tikzpicture}
  \begin{axis}[
    ybar,
    bar width=2pt,
    xticklabels from table={data/ActionCorrectnessTransposed.csv}{Action},
    xtick=data,
    x tick label style={
    rotate=45,
    anchor=east,
    },
    width=0.40\textwidth,
    ymin = 0,
    ymax = 1]
    \addplot table[x=Row, y=6] {data/GestureCorrectness.csv};
    \addplot table[x=Row, y=6] {data/ActionCorrectnessTransposed.csv};
    \addplot table[x=Row, y=6] {data/LocationCorrectness.csv};
  \end{axis}
  \end{tikzpicture}
  \figcaption{Participant 6.}
  \label{fig:participant-2}
\end{longtable}
\caption{Average correctness rate for gestures, actions and locations for the participants who performed the best and worst respectively.}
\end{figure}

\begin{figure}[]
\centering
\begin{tikzpicture}
\begin{axis}[
    ybar,
    bar width=2pt,
    xticklabels from table={data/ActionCorrectnessTransposed.csv}{Action},
    xtick=data,
    x tick label style={
    rotate=45,
    anchor=east,
    },
    width=0.42\textwidth,
    ymin = 0,
    ymax = 1,
    ylabel = correctness rate,
    legend style={
    at={(0,0)},
    anchor=north west, at={(axis description cs:1,0)}}]
    \addplot table[x=Row, y=Average] {data/GestureCorrectness.csv};
    \addplot table[x=Row, y=Average] {data/ActionCorrectnessTransposed.csv};
    \addplot table[x=Row, y=Average] {data/LocationCorrectness.csv};
    \legend{Gesture, Action, Location}
\end{axis}
\end{tikzpicture}
\caption{All participants combined.
Bars in order from left to right:
\emph{Gesture correctness rate}, \emph{Action correctness rate}, \emph{Location correctness rate}}
\end{figure}

\subsection{Conclusion}
\label{sec:evaluation:user-tests-conclusion}

Comparing the results with the requirements specified in\cref{sec:requirements-specification}, only the positioning part succeeded.
The low successrate of actions is likely caused by the low successrate of the gesture but may also be due to the way we have modeled the Bayesian network.
Though the requirements were not fulfilled, the average correctness rates seem promising and with a bit more work done on improving the gesture recognition, and a perhaps the Bayesian network this project seems feasible.

%%% Local Variables:
%%% mode: latex
%%% TeX-master: "../../master"
%%% End:
