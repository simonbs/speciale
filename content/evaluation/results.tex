\section{User Tests}
\label{sec:evaluation:user-tests}

This section will describe the experimental procedures carried out during the project to determine how well it performed when used by
other people and how well it performed comparing to the requirements specified in \cref{}.

\subsection{Setup}
\label{sec:evaluation:user-tests-setup}

The setup of our user test consisted of a Macbook Pro running OpenHAB and Spotify, two Philips Hue lamps and two Estimote beacons.
Seven people were asked to train four unique gestures, creating ten templates per gesture resulting in a total of 40 gesture templates
per person.
The gestures used were:

\begin{itemize}
  \item Circle
  \item Swipe left to right
  \item V
  \item Zorro
\end{itemize}

The template library was cleared before each test so each participant was only using his own templates and not those created by others.

The participants were instructed how to perform the gestures but were allowed to scale them to their personal preference, eg.
either create large circles or small circles.

Once the gesture templates had been created the participants were asked to complete seven tasks, each of which required them to perform a given gesture five times.
Tests took place in a single room but to simulate the user moving between two different rooms, only one Estimote beacon would be turned on at a time.
A list of the gestures and locations is shown in in \cref{table:user-test-tasks}, along with the OpenHAB actions they were supposed to trigger.

\begin{table}[]
\centering
\begin{tabular}{|l|l|l|}
\hline
Action                  & Gesture             & Location                       \\ \hline
Shelves\_Lamp\_Toggle   & V                   & Home Office                    \\ \hline
Spotify\_PlayPause      & Circle              & Home Office                    \\ \hline
Spotify\_Next           & Swipe Left to Right & Home Office                    \\ \hline
Architect\_Lamp\_Toggle & V                   & Living Room                    \\ \hline
TV\_Lamp\_Toggle        & Zorro               & Living Room                    \\ \hline
Spotify\_Next           & Swipe Left to Right & Home Office (Virtual Position) \\ \hline
Shelves\_Lamp\_Toggle   & V                   & Home Office                    \\ \hline
\end{tabular}
\caption{The actions, gestures and locations that were used during the user tests.}
\label{table:user-test-tasks}
\end{table}

\subsection{Results}
\label{sec:evaluation:user-tests-results}

For each participant and task, the correctness ratio of the actions triggered was calculated as the \emph{number of times the intended action was triggered} divided by
the \emph{number of attempts}.
Additionally, the average correctness ratio for each action across all participants as well as the average correctness ratio for each participant across
all actions have been determined and is plotted in \cref{fig:user-test-action-correctness}.

As shown in \cref{fig:user-test-action-correctness}, the correct action was triggered less than 80\% of the time for the majority of the actions.
To determine why this is the case we looked at the rate at which the intended gesture received the highest score.
\cref{fig:user-test-gesture-correctness} shows the correctness ratio of the gestures calculated as the \emph{number of times the intended gesture scored the lowest} divided by
the \emph{number of attempts}.
The results were mixed between the participants with one of them having an average correctness ratio of \number0.925 while another had a ratio of \number0.2.

\begin{figure}
\centering
\begin{tikzpicture}
  \begin{axis}[
      ybar,
      bar width=2pt,
      xlabel = Intended Action,
      ylabel = Correctness ratio,
      xtick=data,
      xticklabels from table={data/ActionCorrectnessTransposed.csv}{Action},
      x tick label style={
      rotate=45,
      anchor=east,
      },
      width=0.95\textwidth,
      height = 6cm,
      yticklabel style={align=right,inner sep=0pt,xshift=-0.3em},
      ymin = 0,
      ymax = 1,
      grid=major,
      try min ticks=10]
    \addplot table[x=Row, y=1] {data/ActionCorrectnessTransposed.csv};
    \addplot table[x=Row, y=2] {data/ActionCorrectnessTransposed.csv};
    \addplot table[x=Row, y=3] {data/ActionCorrectnessTransposed.csv};
    \addplot table[x=Row, y=4] {data/ActionCorrectnessTransposed.csv};
    \addplot table[x=Row, y=5] {data/ActionCorrectnessTransposed.csv};
    \addplot table[x=Row, y=6] {data/ActionCorrectnessTransposed.csv};
    \addplot table[x=Row, y=7] {data/ActionCorrectnessTransposed.csv};
    \addplot table[black, x=Row, y=Average] {data/ActionCorrectnessTransposed.csv};
  \end{axis}
\label{fig:user-test-action-correctness}
\end{tikzpicture}
\caption{The rate at which the correct action was triggered for per participant. The last bar for each action is the average correctness ratio of all participants.
The last group of bars is the average correctness ratio per user, across all actions.}
\end{figure}

\begin{figure}
\centering
\begin{tikzpicture}
\begin{axis}[
    ybar,
    bar width=2pt,
    xticklabels from table={data/GestureCorrectness.csv}{Intended Gesture},
    xtick=data,
    x tick label style={
    rotate=45,
    anchor=east,
    },
    width=0.95\textwidth,
    height = 6cm,
    ymin = 0,
    ymax = 1]
    \addplot table[x=Row, y=1] {data/GestureCorrectness.csv};
    \addplot table[x=Row, y=2] {data/GestureCorrectness.csv};
    \addplot table[x=Row, y=3] {data/GestureCorrectness.csv};
    \addplot table[x=Row, y=4] {data/GestureCorrectness.csv};
    \addplot table[x=Row, y=5] {data/GestureCorrectness.csv};
    \addplot table[x=Row, y=6] {data/GestureCorrectness.csv};
    \addplot table[x=Row, y=7] {data/GestureCorrectness.csv};
    \addplot table[x=Row, y=Average] {data/GestureCorrectness.csv};
\end{axis}
\end{tikzpicture}
\label{fig:user-test-gesture-correctness}
\caption{The correctness ratio for gestures. The last bar for each gesture is the average correctness ratio of all participants.
The last group of bars is the average correctness ratio per user, across all gestures.}
\end{figure}

\begin{figure}
\centering
\begin{tikzpicture}
\begin{axis}[
    ybar,
    bar width=2pt,
    xticklabels from table={data/LocationCorrectness.csv}{Action},
    xtick=data,
    x tick label style={
    rotate=45,
    anchor=east,
    },
    width=0.95\textwidth,
    height = 6cm,
    ymin = 0,
    ymax = 1]
    \addplot table[x=Row, y=1] {data/LocationCorrectness.csv};
    \addplot table[x=Row, y=2] {data/LocationCorrectness.csv};
    \addplot table[x=Row, y=3] {data/LocationCorrectness.csv};
    \addplot table[x=Row, y=4] {data/LocationCorrectness.csv};
    \addplot table[x=Row, y=5] {data/LocationCorrectness.csv};
    \addplot table[x=Row, y=6] {data/LocationCorrectness.csv};
    \addplot table[x=Row, y=7] {data/LocationCorrectness.csv};
    \addplot table[x=Row, y=Average] {data/LocationCorrectness.csv};
\end{axis}
\end{tikzpicture}
\label{fig:user-test-location-correctness}
\caption{The correctness ratio for locations. The last bar for each action is the average correctness ratio of all participants.
The last group of bars is the average correctness ratio per user, across all gestures.}
\end{figure}

%%% Local Variables:
%%% mode: latex
%%% TeX-master: t
%%% End:
