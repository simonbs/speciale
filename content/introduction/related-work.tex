\section{Related Work}
\label{sec:related-work}

The following section presents the work of others that relates to the work described in this report as well as the differences between the two.

Various ways of interacting with the systems in a smart home are presented in \cite[pp. 9-10]{cook2007smart} including speech, facial expressions and gestures. The latter of which has been found to be an easy way of interacting with systems \cite[p. 6]{rahman2011motion} and is utilized in the solution presented in this report. In \cite[pp. 2-3]{starner2000gesture} motion gestures are described to be more convenient than regular remotes, as they require the user to be carrying the remote with them or in the case of wall mounted panels, walk up to the remote. Furthermore the paper describes how speech commands may drown in the noise if users are controlling a media center and that interaction using speech is inconvenient in multi-user configurations.

In \cite[pp. 9-11]{prespecialisation} we described Reemo as related work. Reemo is a solution in which users point at devices and control them using motion gestures \cite{reemo:about}. While the company has not released any details about their technology, we can see from their website that the solution is limited to devices within line of sight as receiver must be placed next to each controllable device.

In \cite{caon2011context} a solution for recognizing context-aware motion gestures using multiple Kinects is presented. Users control devices by pointing at them and depending on the current state of the device and the position of the user, different actions are triggered.
The authors use two Kinects to position the user and recognize motion gestures in a living room, requiring a total of 6-8 Kinects in an appartment consisting of 3-4 rooms.

\subsection{Conclusion}

The solution presented in this report differs from the above solutions in the way, that controllable devices are not required to be within line of sight. Users of our solution are able to control all devices within the system froom anywhere in their house.

Furthermore the solution will utilize beacons for positioning users as opposed to Kinects as utilized in \cite{caon2011context}. This lowers entry barrier by decreasing the initial cost as well as the cost for scaling the system to more rooms.

Gestures are used for controlling the smart home as they are convenient and by utilizing the accelerometer, a common component in wearables \cite[pp. 3-4]{prespecialisation}, the user may not need to buy hardware specifically for recognizing gestures is he owns a wearable with an accelerometer and this wearable is supported in the system.


%%% Local Variables:
%%% mode: latex
%%% TeX-master: "../../master"
%%% End:
