\section{Target Group}
\label{sec:target-group}

In \cite[p. 15]{prespecialisation} we defined the target group as people living in a smart home and interested in controlling the state of their home, \eg~lights, music centers, doors and windows, in a convenient matter. The group of people will currently consist of early adoptors of smart home technologies but based on the trend in IoT, our assumption is that the technology will be widespread within 5-10 years. 

We extend the definition of the target group presented in \cite[p. 15]{prespecialisation} to include a definition of the surroundings. As shown in \cref{appendix:housing-types:table} the most common size of housings in Denmark is 75-99 square meters. Therefore we assume that the solution presented in this report is installed in a typical Danish apartment with a size of 90 square meters and with 3-4 rooms. 

We assume each room has two-three controllable lamps. The living room may have a television and a music center while the kitchen may have a controllable coffee machine. We estimate that 2-5 controllable devices per room seems fair leaving us with a maximum of 20 controllable devices in an apartment.

%%% Local Variables:
%%% mode: latex
%%% TeX-master: "../../master"
%%% End:
