\section{Problem Statement}
\label{sec:problem-statement}

In our previous report we formulated the hypothesis that wearables can be integrated into home automation environments to provide an interface for controlling devices that belong to the concept of Internet of Things \cite[pp. 14]{prespecialisation}.

In the previous report \cite[pp. 69-73]{prespecialisation} we found that motion gestures can be used to control a home automation environment. We found the system to have an accuracy of 4.29\%, i.e. the the correct action was performed 4.29\% of the time. The poor accuracy was due to the use of fine grained position information used when determining which device in the smart home to control.

The position of the user is utilized when determining which device the user points at and thus which device should be controlled. The future work of the report \cite[pp. 71-73]{prespecialisation} suggests using contextual information to determine which device should be controlled rather than the granular positioning. By doing this it is no longer possible for the user to point at a device in order to control it but we may be able to improve the accuracy of the solution and given the correct contextual information we can narrow the set of devices the user desires to control sufficiently to provide an attractive solution for controlling a smart home.

The solution described in our previous report was implemented on an Android smartphone due to time restrictions. The future work section of the report suggests implementing the solution on a wearable device.

It is our hypothesis that we can utilize contextual information to determine which device a user intends to control in a smart home environment. Our problem statement is as follows.

\begin{framed}
  How can we utilize contextual information when controlling a smart home using a wearable in a gesture driven solution?
\end{framed}

%%% Local Variables:
%%% mode: latex
%%% TeX-master: "../../master"
%%% End:
