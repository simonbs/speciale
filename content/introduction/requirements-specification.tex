\section{Requirements Specification}
\label{sec:requirements-specification}

This section presents the requirements for the solution. The requirements are divided into the following three groupings.

\begin{description}
\item[Functional requirements] These represent the functionality the solution should implement.
\item[Performance requirements] These requirements describe how well the solution should perform in given conditions.
\item[Overall requirements] These are requirements that the system as a whole should fulfill and that does not fit within any of the two other groupings.
\end{description}

\subsection{Functional Requirements}

\begin{description}
\item[Train and recognize gestures] Users should be able to train a motion gesture and the system should be able to recognize the gesture when the user performs this gesture.
\item[Trigger actions on controllable devices using gestures] Users should be able to trigger an action on a controllable device by performing a gesture using a wearable.
% \item[Receive feedback if a gesture is not recognized] If a gesture is not recognized, the user should receive feedback. If possible, the system should guess which gestures the user was likely to perform and suggest the actions associated with the guessed gesture to the user.
\item[Context-aware] The system should be context-aware such that different actions may be triggered from the same gesture depending on contextual information. For example, a circular gesture may turn on the TV when the user is in the living room but turn on the lights when the user is in the kitchen.
\item[Associate a gesture with actions] The user should be able to associate a gesture with one or more actions that a controllable device can perform. A gesture can only be associated with multiple actions, if the controllable devices to which the action belongs reside in different rooms. Actions change the state of one or more smart devices in a smart home when triggered.
\item[Virtual positioning of users] A user should be able to virtually position himself in his home using the wearable. This allows the user to perform gestures in one room, while being in another.
\end{description}

\subsection{Performance Requirements}

\begin{description}
\item[Limit the amount of gestures by letting devices share gestures] As described in \Cref{sec:introduction:gesture-control}, users should be able to use the same gestures for multiple devices and thus reduce the overall amount of gestures they need to recall.
\item[Handle 15-20 controllable devices] As described in \Cref{sec:analysis:scenarios} we assume the system is deployed in an apartment with 3-4 rooms with 2-5 devices in each room. Therefore the system should be able to handle a minimum of 15-20 controllable devices.
\item[Trigger Correct Action] The correct action should be triggered at least 80\% of the time. An action is considered correct if it is the one that the user intended.
\end{description}

\subsection{Overall Requirements}

\begin{description}
\item[Use inexpensive hardware and software] As described in \Cref{sec:analysis:scenarios} the target group are people living in a smart home, typically in an apartment with a size of 90 square meters. Therefore we are not interested in expensive solutions when looking at hardware or software.
\item[Not limited to smart devices in line of sight] Reemo, a related solution, limits the user to control devices that are in line of sight. We do not want to have this limitation in our solution, as it limits the smart devices that can be controlled.
\end{description}

%%% Local Variables:
%%% mode: latex
%%% TeX-master: "../../master"
%%% End:
