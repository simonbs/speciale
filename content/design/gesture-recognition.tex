\section{Gesture Recognition}
\label{sec:design:gesture-recognition}

As stated in \Cref{sec:analysis:choice-of-gesture-recognizer} we decided to use the 1\textcent~gesture recognizer.
However, as the 1\textcent~gesture recognizer is designed to be used for recognition of two-dimensional gestures we have created a modified version of it that is able to recognize three-dimensional gestures recorded using a three-axis accelerometer by utilizing one of the techniques used by \$3~\cite{threedollar}.

\subsection{Converting 1\textcent~to Three Dimensions}
For use in this project we created a modified version of 1\textcent~that works with three-dimensional accelerometer data.
A naive approach would be to use raw accelerometer data as coordinates but the problem with this approach, besides lack of filtering, is that these coordinates would not properly represent a gesture.

Imagine a minimalistic example where the the accelerometer registers two consecutive accelerations, the first one is on the $x$-axis and the other is on the $y$-axis.
The $z$-axis is ignored for the sake of the argument.
This is illustrated by the two blue arrows in \Cref{fig:accelerometerpoints}.
If these two measurements are used as coordinates for the gesture, it would end up looking like the gesture shown in \Cref{fig:accelerometerpath}.

\begin{figure}[h]
\centering
% Pure accelerometer points
\begin{tikzpicture}
  \draw[gray, ->, thick] (0,-5) -- (0,5) node[left] {$y$};
  \draw[gray, ->, thick] (-5,0) -- (5,0) node[below] {$x$};
  \draw[blue, ->, thick] (0,0) -- (4,0);
  \draw[blue, ->, thick] (0,0) -- (0,4);
  \filldraw[red] (0,0) circle (2pt);
\end{tikzpicture}
\caption{Two consecutive accelerometer measurements $\{4,0\}$ and $\{0,4\}$. Accelerations are measured in $\frac{m}{s^2}$}
\label{fig:accelerometerpoints}
\end{figure}

\begin{figure}[h]
    \begin{subfigure}{.5\linewidth}
    \resizebox{\linewidth}{!}{
        % Pure accelerometer path
        \begin{tikzpicture}
            \draw[gray, ->, thick] (0,-5) -- (0,5) node[left] {$y$};
            \draw[gray, ->, thick] (-5,0) -- (5,0) node[below] {$x$};
            \draw[black, thick] (0,0) -- (4,0) -- (0,4);
            \filldraw[red] (0,0) circle (2pt);
            \filldraw[black] (4,0) circle (2pt);
            \filldraw[black] (0,4) circle (2pt);
        \end{tikzpicture}
    }
    \caption{Path drawn between the measured points by using measurements as coordinates.}
\label{fig:accelerometerpath}
    \end{subfigure}
    \begin{subfigure}{.5\linewidth}
    \resizebox{\linewidth}{!}{
        % 3$ path
        \begin{tikzpicture}[scale=0.6]
        \draw[gray, ->, thick] (0,-5) -- (0,5) node[left] {$y$};
        \draw[gray, ->, thick] (-5,0) -- (5,0) node[below] {$x$};
        \draw[black, thick] (0,0) -- (4,0) -- (4,4);
        \filldraw[red] (0,0) circle (2pt);
        \filldraw[black] (4,0) circle (2pt);
        \filldraw[black] (4,4) circle (2pt);
        \end{tikzpicture}
    }
    \caption{The proper path of the accelerometer-recorded gesture by adding accelerations together.}
\label{fig:acceleromterpath-proper}
    \end{subfigure}
\caption{Minimalistic example of the difference between using accelerometer measurements directly as coordinates in a gesture trace compared to adding accelerations together as in \$3.}
\end{figure}

The proper path is illustrated in \Cref{fig:acceleromterpath-proper}.
The way we achieve the proper path is by borrowing a technique from~\cite{threedollar} where points are acquired by subtracting the current accelerometer measurement from the previous one to obtain an acceleration delta and adding it to the previous point to create the new point as shown in \Cref{eq:acceleration-timeseries}.
However, in our specific implementation we do not add the acceleration deltas to the previous point but instead add the current acceleration measurement.

%%% Local Variables:
%%% mode: latex
%%% TeX-master: "../../master"
%%% End:
