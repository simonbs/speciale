\section{Gesture Context Provider}
\label{sec:design:gesture-context-provider}

The following section describes the gesture context provider introduced in \cref{sec:analysis:context-engine}.

The gesture context provider determines the actions it makes sense to trigger based on the motion performed by the user. When the user performs a motions, the gesture recognizer scores the trained gestures.

When the gestures are scored, they are passed to the context provider which calculates the probabilities of each action associated with the gesture. The calculation of probabilities is similar to the calculation of probabilities for the position context provider presented in \cref{sec:design:position-context-provider}.

When calculating the probabalities, all actions associated to each gesture is retrieved, the scores are normalized and distributed to each action.

For example, if gestures A and B are recognized with A having a score of 47 and B having a score of 93, the total score is $47 + 93 = 140$. Assume that A is associated with actions 1 and 2 and B is associated with action 3.

The lower the score is, the better. Therefore the probability of gesture A being recognized is considered to be the following.

\begin{equation*}
(1 - \frac{47}{140}) \cdot 100 = 66.43
\end{equation*}

The probability of gesture B is considered to be the following.

\begin{equation*}
(1 - \frac{93}{140}) \cdot 100 = 33.57
\end{equation*}

The probabilities are assigned to the actions as weights. The weights are then normalized as shown in \cref{tbl:sec:design:gesture-context-provider:weighted-actions}. The weights are passed to the context recognizer.

\begin{table}[h!]
\centering
\caption{Example of weights assigned to actions.}
\label{tbl:sec:design:gesture-context-provider:weighted-actions}
\begin{tabular}{lll}
                  & \textbf{Weight} & \textbf{Normalized weight} \\
\textbf{Action 1} & 66.43           & 66.43 / 166.43             \\
\textbf{Action 2} & 66.43           & 66.43 / 166.43             \\
\textbf{Action 3} & 33.57           & 33.57 / 166.43             \\
\textbf{Total}    & 166.43          & 1                         
\end{tabular}
\end{table}

%%% Local Variables:
%%% mode: latex
%%% TeX-master: "../../master"
%%% End:
