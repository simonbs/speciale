\section{Recommender System Methods}

\subsection{Collaborative Filtering}
\label{sec:analysis:recommender-methods:collaborative-filtering}

Collaborative filtering is technique that uses information about multiple users to recommend items for a given user.
In a setting where it is used to recommend movies a source of information could be user-submitted movie ratings.
Based on the genres and actors of the movies that a given user has rated highly, the recommender sytem begins acquiring the preferences of that user.
It would  then find users with similar preferences and look at which movies they have rated highly as they may then be relevant to a given user.

Some of the issues with collaborative filtering according to Claypool et.~al~\cite{claypool1999combining} are that it depends heavily on users actually providing ratings, otherwise the recommender system will have nothing to base it recommendations on.
This issue will occur every time a new item is added as users have yet to rate it.
In addition when a new recommender system created it will suffer from lack of data and every user will be afflicted by this ``Early rater problem'' for all items.

% We deem collaborative filtering unsuited for the purposes of this project as it requires multiple comparable users and our project is tailored more towards individual users.

\subsection{Content-based Filtering}
\label{sec:analysis:recommender-methods:content-based-filtering}

A different method that Claypool et.~al utilize to compensate for the ``Early rater problem'' is content-based filtering which relies on the correlation between a users preferences and the information about items.
Eg.~a user who likes to read news articles regarding specific topics is more likely to be presented with articles with such from a recommender system using content-based filtering, regardless of how other users have rated that article.
One of the challenges of this technique is to create proper classification of items and their attributes along with user profiles such that they can be matched.
Pazzani et.~al~\cite{pazzani1996syskill} classify web pages by analysing the word frequencies, excluding common English words.
To determine the preferences of the user some form of relevancy feedback is required, either positive or negative.
Pazanni et.~al have the user categorize the web pages he visits as either \texttt{hot}, \texttt{luke-warm} or \texttt{cold}.
It is also possible to use implicit feedback, as discussed by Meteeren et.~al~\cite{van2000using}, such as the time spent on a web page.
However when using implicit feedback one has to be cautious with how strong belief is put into each observation as with Meteeren et.~al they mention that one could look at ignored links as being negative feedback on those pages, but perhaps the user did not dislike the link but rather did not see it.

%We deem content-based filtering unfit for this project as it would be complicated to come up with attributes that would fit the items in our recommender system.

%Perhaps add a section about knowledge-based or hybrid systems

\subsection{Decision Trees}
\label{sec:analysis:recommender-methods:decision-trees}

Decision trees are a model for decision making and prediction using a graph structure where leaf nodes represent outcome and non leaf nodes test attributes or random variables.
The branches that leave non leaf nodes are labeled with the value of the corresponding attribute.
Decision trees need not be balanced trees so some paths may end in a leaf node after very few nodes, as in figure 2 of Cho et.~al~\cite{cho2002personalized} where if the root node evaluates to ``manager'' the tree immediately ends in a leaf node ``buy''.
Decision trees are commonly used for content-based filtering~\cite{adomavicius2005toward}.

\subsection{Bayesian Networks}
\label{sec:analysis:recommender-methods:bayesian-networks}

Bayesian networks are a model for for prediction represented as a directed acyclic graph where each node represents a random variable and edges between represent conditional dependencies.
This means that any child node is conditionally dependent on its parents.
Each node or item in the network has a probability distribution that may change when evidence is observed.
A node representing the probability for a given disease may be conditionally dependent in such a way that observing the symptoms of a patient will change the likelihood that the patient is afflicted.
One of the advantages of bayesian networks is that they can handle missing data well~\cite{heckerman2008tutorial}.

% \subsection{Neural Network}
% \label{sec:analysis:recommender-methods:neural-networks}
% These take too long to train

\subsection{Conclusion}
\label{sec:analysis:recommender-methods:conclusion}

A multitude of different solutions have been proposed for recommender systems~\cite{adomavicius2005toward} with just a few presented in this report.
Collaborative filtering, presented in~\cref{sec:analysis:recommender-methods:collaborative-filtering}, can be a useful approach when different users are comparable and the interests of other users may have a positive impact on a given user.
Rather we think that content-based filtering may prove a better fit for this project.
However in this project we focus on single user needs and do not see the behaviour of other users as applicable to a given user.
Decision trees, described in~\cref{sec:analysis:recommender-methods:decision-trees}, though prevalent for use in content-based recommender systems, do not handle missing data as well as bayesian networks and thusly we deem bayesian networks a better suited model for the purposes of this project.