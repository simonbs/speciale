\section{Choice of Hub}
\label{sec:analysis:choice-of-hub}

As mentioned in \Cref{sec:analysis:scenarios}, we intend to use a central hub for connecting devices in the smart home setting. In~\cite{prespecialisation} we presented a list of different home automation hubs and decided to use HomePort.

For this project we decided to revisit the following hubs to determine which one we deem to be the right one to use now.

\begin{itemize}
    \item HomePort
    \item SmartThings
    \item OpenHAB
\end{itemize}

Common for all three solution is that they serve as adapters. Devices that fit within the concept of IoT, \eg~smart bulbs, media centres, windows, door locks \etc~communicate using different protocols. These protocols include ZigBee, Z-Wave, HTTP and Bluetooth. The hubs expose devices communicating using those protocols under a common interface \eg~HTTP.

The hub should fulfill the following requirements.

\begin{itemize}
\item We intend to keep the price of the entire solution presented in this report at a minimum, therefore we are interested in an inexpensive hub.
\item As the primary focus of the report is not on the development and research of a hub, we intend to choose a solution which can easily be used within our system with little to no modifications.
\item The hub should be commonly accessible.
\end{itemize}

\subsection{HomePort}
\label{sec:analysis:choice-of-hub:homeport}

HomePort is a free open and source software solution developed as a research project at the Institute of Computer Science at Aalborg University~\cite{HOMEPORT10,homeport:github}.
HomePort is a software solution that the user needs to install on a computer that will then act as the hub.

HomePort supports communication with Phidget USB devices and expose them over the HTTP protocol. Other applications can control the Phidget devices and read their state by issuing requests to HomePort.

When using HomePort in~\cite{prespecialisation} we experienced the following issues.

\begin{itemize}
\item We found that the Phidget adapter was not compatible with the most recent version of HomePort.
\item We had difficulties getting an older version of HomePort with a compatible Phidget adapter running on OS X.
\item We got the older version of HomePort running on a Windows machine but found that it would frequently crash requiring a reboot of the HomePort software.
\end{itemize}

\subsection{SmartThings}
\label{sec:analysis:choice-of-hub:smartthings}

SmartThings~\cite{SMARTTHINGS}\footnote{More information about SmartThings is available at \url{https://www.smartthings.com/products/hub}} is commercial home automation hub offered by Samsung. SmartThings is a hardware device that comes preinstalled with the necessary software.

Protocols supported by SmartThings include ZigBee and Z-Wave as well as devices communicating over WiFi within the local network~\cite{smartthings:compatible-products}. SmartThings expose the devices over HTTP using a REST API~\cite{smartthings:developer}.

While exposing devices using a REST interface, Samsung also provides a concept of SmartApps, applications installed in the cloud, \ie~on Samsungs servers. These applications react to events within the users SmartThings environment, \ie~a new temperature reading and triggers other events based on the new information. SmartApps can also run events periodically.

At the time of writing the retail price for the hub is \$99.

\subsection{OpenHAB}
\label{sec:analysis:choice-of-hub:openhab}

Unlike SmartThings, openHAB~\cite{OPENHAB} does not come with the hardware needed to run the hub.
openHAB is based on Eclipse SmartHome\footnote{The Eclipse SmartHome project can be found at \url{https://www.eclipse.org/smarthome/}}, an open source framework for building software for a smart home. Amongst others, the framework provides mechanisms for handling data (including a common type system) and a rule engine.
openHAB and Eclipse SmartHome are both Java based software solutions that support any platform capable of running a JVM~\cite{openhab:introduction}.

openHAB is free and open source maintained by a community of enthusiasts. It supports a wide range of devices, 121 as of writing~\cite{openhab:supported-technologies}. Amongst these are devices communicating using ZigBee, Z-Wave, MQTT and HTTP. openHABs concept of adapters is called ``bindings'' and these can be specific to a protocol or a product. For example, the community has developed bindings for the HTTP, Bluetooth and MQTT protocols as well as for the Philips Hue, Netatmo, and Sonos products.

The devices connected to an openHAB environment is exposed over HTTP.

\subsection{Conclusion}
\label{sec:analysis:choice-of-hub:conclusion}

While the SmartThings hub is a consumer oriented solution which is easy to install, we want to keep the price of our solution to a minimum and thus prefer a free software solution rather than purchase the SmartThings hub.

The choice between HomePort and openHAB came down to the following two things.

\begin{itemize}
\item We experienced various issues with HomePort. In a previous project~\cite{prespecialisation} we found that the Phidget adapter is incompatible with the most recent version of HomePort, that we were unable to run previous a previous version on OS X and that the same version would frequently crash on Windows.
\item Furthermore we found that the community developing openHAB has created bindings for hardware we already have access to and assuming these work, we will not need to create our own adapters.
\end{itemize}

The main focus of this report is not the research or development of a smart hub and therefore we do not intend to spend much time installing and configuring a hub. Hence we think that openHAB is the best fit for this project and will be our choice of hub.

%We will run openHAB on a Raspberry Pi as the price of the computer is low reducing the barrier of entry to the system and it is a very small computer making it easy to hide it in \eg~a closet.

\subsection{Hardware}

The hardware that, together with openHAB, constitutes the hub should fulfill the following requirements.

\begin{itemize}
\item Compatible with openHAB as this is our software of choice.
\item Low cost in order to reduce the barrier of entry.
\item Small in order for it to easily fit into any place of a home, \eg~a closet.
\end{itemize}

openHAB runs on any computer that can run a Java virtual machine (JVM)~\cite{openhab:introduction}. It can be installed on desktop computers running Windows, Linux or OS X.

Typical desktop computers are not a great fit for this project as they are typically large and take up more space than recently introduced tiny computers such as the Raspberry Pi. The tiny computers are typically assembled from hardware components that are less powerful than the components installed in desktop computers. Therefore tiny computers consume less power than traditional desktop computers thus reducing the long-term cost of running the hub.

Furthermore tiny computers such as the Raspberry Pi Zero are priced as low as \$5 dollars~\cite{raspberrypi:zero}. While the Pi Zero requires an adapter to be connected to the Internet, the total cost of the solution is still less than a setup using a desktop computer. Recently Adafruit, the company behind the Raspberry Pi, released the Raspberry Pi 3, which is the first Raspberry Pi with built-in WiFi and BLE.

Raspberry Pi is completely compatible with openHAB and is one of the recommended computers for running openHAB~\cite{openhab:hardware}. Therefore we choose to base the hub on a Raspberry Pi but the hub should be able to run on any machine that can run a JVM.

%%% Local Variables:
%%% mode: latex
%%% TeX-master: "../../master"
%%% End:
