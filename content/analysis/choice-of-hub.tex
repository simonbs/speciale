\section{Choice of Hub}
\label{sec:analysis:choice-of-hub}

As mentioned in \cref{sec:analysis:scenarios}, we intend to use a central hub for connecting devices in the smart home setting. Which hub will be used will be explained in this section.

In our previous report \cite{previous-report} we presented a list of different home automation hubs and decided to use HomePort.

We have decided to revisit the following hubs to determine which one we deem to be the right one to use now:

\begin{itemize}
    \item HomePort
    \item SmartThings
    \item OpenHAB
\end{itemize}

\subsection{HomePort}
\label{sec:analysis:choice-of-hub:homeport}

Homeport \cite{HOMEPORT10,homeport:github} is a free open source software solution developed as a research project at the Institute of Computer Science at Aalborg University.
Since HomePort is a software solution the user needs to install it on a computer that will then act as the hub.

As mentioned in \cref{sec:analysis:choice-of-hub}, HomePort was the one we chose to use in our previous project. However, there were some difficulties getting it to work on our machines.

\subsection{SmartThings}
\label{sec:analysis:choice-of-hub:smartthings}

SmartThings \cite{SMARTTHINGS} is product developed by Samsung\todo[author=Kasper]{Er faktisk lidt i tvivl om det her, tror måske det blev udviklet af nogle andre og så opkøbt af Samsung.} but unlike HomePort, described in \cref{sec:analysis:choice-of-hub:homeport}, the user does not need to install the hub software on his own computer but instead purchases a physical hub\footnote{https://www.smartthings.com/products/hub} and connects it to his home network.
At the time of writing the retail price for the hub is \$99.

\subsection{OpenHAB}
\label{sec:analysis:choice-of-hub:openhab}

OpenHAB \cite{OPENHAB} is, much like HomePort, a free open source software solution.
However it is not a research project but is instead community driven with 10 official maintainers \cite{openhab:maintainers}.
As of writing it has a list of 121 supported devices \cite{openhab:supported-technologies}.

\subsection{Conclusion}
\label{sec:analysis:choice-of-hub:conclusion}

Although the SmartThings hub presented in \cref{sec:analysis:choice-of-hub:smartthings} is easy to install for the user, we want to keep the price of our solution to a minimum and thus would rather go for a free software solution rather than purchase the SmartThings hub.

The choice between HomePort and OpenHAB came down to two things.

First off we have tried using Homeport once before in \cite{previous-report} and it was not very easy for us to set up and use.
As well there are only a few people actively developing it so there is only support for a few smart devices.

The other thing is that, due to the amount of people working on, as well as using, OpenHAB compared to HomePort, there are more people that can assist us when issues arise.

Hence we think that OpenHAB is the best fit for this project and will be our choice of hub.

\subsection{Hardware}

The hardware that, together with OpenHAB, constitutes the hub should fulfill the following requirements.

\begin{itemize}
\item Compatible with OpenHAB as this is our software of choice.
\item Low cost in order to reduce the barrier of entry.
\item Small in order for it to easily fit into any place of a home, e.g. a closet.
\end{itemize}

OpenHAB runs on any computer that can run a Java virtual machine (JVM) \cite{openhab:introduction}. It can be installed on desktop computers running Windows, Linux or OS X.

Typical desktop computers are not a great fit for this project as they are typically large and take up more space than recently introduced tiny computers such as the Raspberry Pi. The tiny computers are typically assembled from hardware components that are less powerful than the components installed in desktop computers. Therefore tiny computers consume less power than traditional desktop computers thus reducing the long-term cost of running the hub.

Furthermore tiny computers such as the Raspberry Pi Zero are priced as low as \$5 dollars \cite{raspberrypi:zero}. While the Pi Zero requires an adapter to be connected to the Internet, the total cost of the solution is still less than a setup using a desktop computer.

Raspberry Pi is completely compatible with OpenHAB and is one of the recommended tiny computers for running OpenHAB \cite{openhab:hardware}. Therefore we choose to base the hub on a Raspberry Pi but the hub should be able to run on any machine that can run a JVM.

%%% Local Variables:
%%% mode: latex
%%% TeX-master: "../../master"
%%% End:
