\subsection{Bayes Theorem}

Up until now we have used \Cref{fig:analysis:bayesian-network:abc} as an example Bayesian network. A more realistic network is illustrated in \Cref{fig:analysis:bayesian-network:sample} where $K = \{Fraud, Age, Sex, Gas, Jewelry\}$, $E = \{\{Fraud, Gas\}, \{Fraud, Jewelry\}, \{Age, Jewelry\}, \{Sex, Jewelry\}\}$ and P can be defined as in \Cref{tbl:analysis:bayesian-network:sample-tables-1,tbl:analysis:bayesian-network:sample-tables-2,tbl:analysis:bayesian-network:sample-tables-3}. The network models credit card fraud. From the network we can see, that if there is credit card fraud then there is a chance of gas or jewelry being bought. The chance of jewelry being bought is also affected by the age and the sex of the purchaser.

\begin{figure}[h!]
\centering
\includegraphics[width=0.5\textwidth]{images/sample_bayesian_network}
\caption{Sample Bayesian network. The network consist of five nodes, i.e. five variables. The Fraud node influence the Gas and Jewelry nodes and the Jewelry node is further influenced by the Sex and Age nodes~\cite{stephenson2000introduction}.}
\label{fig:analysis:bayesian-network:sample}
\end{figure}

\begin{table}[h!]
\centering
\caption{Probability tables for the Fraud, Gas and Sex nodes in the network illustrated in \Cref{fig:analysis:bayesian-network:sample}. Values are from~\cite{stephenson2000introduction}.}
\label{tbl:analysis:bayesian-network:sample-tables-1}
\begin{tabular}{ccc}
\begin{tabular}{c}
\textbf{Fraud}   \\
\begin{tabular}{cc}
Yes   & No \\ \hline
0.00001 & 0.99999
\end{tabular}
\end{tabular}
&
\begin{tabular}{c}
\textbf{Age}   \\
\begin{tabular}{ccc}
< 30 & 30-50 & > 50  \\ \hline
0.25 & 0.4 & 0.35
\end{tabular}
\end{tabular}
&
\begin{tabular}{c}
\textbf{Sex}   \\
\begin{tabular}{cc}
Male   & Female \\ \hline
0.5    & 0.5
\end{tabular}
\end{tabular}
\end{tabular}
\end{table}

\begin{table}[h!]
\centering
\caption{Conditional probability table for the Gas node in the network illustrated in \Cref{fig:analysis:bayesian-network:sample}. Values are from~\cite{stephenson2000introduction}.}
\label{tbl:analysis:bayesian-network:sample-tables-2}
\begin{tabular}{c}
\textbf{Gas}   \\
\begin{tabular}{l|cc}
             & Gas = Yes   & Gas = No \\ \hline
Fraud = Yes  & 0.2         & 0.8 \\
Fraud = No   & 0.01       & 0.99 \\
\end{tabular}
\end{tabular}
\end{table}

\begin{table}[h!]
\centering
\caption{Conditional probability table for the Jewelry node in the network illustrated in \Cref{fig:analysis:bayesian-network:sample}. Values are from~\cite{stephenson2000introduction}.}
\label{tbl:analysis:bayesian-network:sample-tables-3}
\begin{tabular}{c}
\textbf{Jewelry} \\
\begin{tabular}{l|c}
Age           & \begin{tabularx}{0.8\textwidth}{Y|Y|Y} < 30 & 30 - 50 & > 50 \end{tabularx} \\ \hline
Fraud         & \begin{tabularx}{0.8\textwidth}{YY|YY|YY} Yes & No & Yes & No & Yes & No \end{tabularx} \\ \hline
Sex           & \begin{tabularx}{0.8\textwidth}{YY|YY|YY|YY|YY|YY} M & F & M & F & M & F & M & F & M & F & M & F \end{tabularx} \\ \hline
Jewelry = Yes & \begin{tabularx}{0.8\textwidth}{YY|YY|YY|YY|YY|YY} $\frac{1}{20}$ & $\frac{1}{20}$ & $\frac{1}{10000}$ & $\frac{1}{2000}$ & $\frac{1}{20}$ & $\frac{1}{20}$ & $\frac{1}{2500}$ & $\frac{1}{500}$ & $\frac{1}{20}$ & $\frac{1}{20}$ & $\frac{1}{5000}$ & $\frac{1}{1000}$ \end{tabularx} \\
Jewelry = No  & \begin{tabularx}{0.8\textwidth}{YY|YY|YY|YY|YY|YY} $\frac{19}{20}$ & $\frac{19}{20}$ & $\frac{9999}{10000}$ & $\frac{1999}{2000}$ & $\frac{19}{20}$ & $\frac{19}{20}$ & $\frac{2499}{2500}$ & $\frac{499}{500}$ & $\frac{19}{20}$ & $\frac{19}{20}$ & $\frac{4999}{5000}$ & $\frac{999}{1000}$ \end{tabularx}
\end{tabular}
\end{tabular}
\end{table}

Given the network illustrated in \Cref{fig:analysis:bayesian-network:sample} we could be interested in computing $P(Fraud|Jewelry = Yes, Gas = No, Sex = Male, Age = < 30)$, that is, the probabilities of fraud happening when we have observed, that a male purchaser younger than 30 years have bought jewelry and have not bought gas.

We compute this using Bayes' theorem which is defined as:

\begin{equation}
P(A|B) = \frac{P(B|A)P(A)}{P(B)}
\end{equation}

More specifically, we want to compute the following. Note that for readability purposes the names of the variables and the states have been shortened to their initial letter.

\begin{equation*}
P(F=Y|J = Y, G = N, S = M, A = < 30)
\end{equation*}
\begin{equation*}
\begin{split}
&= \frac{P(J = Y, G = N, S = M, A = < 30|F=Y) \cdot P(F=Y)}{P(J = Y, G = N, S = M, A = < 30)} \\
&= \frac{P(J = Y, S = M, A = < 30|F=Y) \cdot P(G=N|F=Y) \cdot P(F=Y)}{P(J = Y, G = N, S = M, A = < 30)}
\end{split}
\end{equation*}

According to \Cref{eq:analysis:bayesian-network:prod} we can write this as

\begin{equation*}
P(F=Y|J = Y, G = N, S = M, A = < 30)
\end{equation*}
\begin{equation*}
= \frac{P(J = Y, S = M, A = < 30|F=Y) \cdot P(G=N|F=Y) \cdot P(F=Y)}{P(J=Y|F=Y,S=M,A=<30) \cdot P(G=N|F=Y) \cdot P(S=M) \cdot P(A=<30)}
\end{equation*}

Then we substitute with the probabilities from our probability tables and conditional probability tables.

\begin{equation*}
P(F=Y|J = Y, G = N, S = M, A = < 30)
\end{equation*}
\begin{equation*}
= \frac{\frac{1}{20} \cdot 0.8 \cdot 0.00001}{\frac{1}{20} \cdot 0.8 \cdot 0.5 \cdot 0.25} =
\end{equation*}
%%% Local Variables:
%%% mode: latex
%%% TeX-master: "../../master"
%%% End:
