\section{Context}
\label{sec:analysis:context}

In the envisioned system the context is used to determine which actions should be triggered when the user performs a gesture. Therefore the context plays an important part in the envisioned system.

The notion of context is researched in multiple fields, including philosophy and psychology~\cite{bolchini2007data}. For the purpose of this project we focus on the notion of context in context-aware software systems. In~\cite{abowd1999towards}, the context for a context-aware system is defined as in \Cref{def:context}.

\begin{definition}
``We have defined context to be any information that can be used to characterize the situation  of  an  entity,  where  an entity  can  be  a  person,  place,  or  object.  In  context-aware  computing,  these  entities  are  anything  relevant  to  the  interaction  between  the user and application, including the user and the application.''~\cite{abowd1999towards}
\label{def:context}
\end{definition}

According to this definition, information is context if the information characterizes the situation of a participant in an interaction~\cite{abowd1999towards}.

Consider the following example.

\begin{testexample}
A clothing store has a system installed that sends notifications to users mobile devices with offers when they are near the clothes that the offer apply to. If a t-shirt is 20\% off, and the user is near the t-shirt, the user will receive a notification letting them know that the t-shirt is on sale.
\end{testexample}

In the above example, context includes:

\begin{itemize}
\item The position of the user.
\item The position of the t-shirt on sale.
\item The sex of the user.
\item The age of the user.
\item The percentage the price of the t-shirt is reduced with.
\end{itemize}

Information such as what other customers are in the store and the time of the day is not context because it is not relevant to the interaction between the user and the application.

\subsection{Context Types}

In~\cite{abowd1999towards} the context is divided into two categories, \emph{primary} and \emph{secondary}. The ``Location, identity, time, and activity ''~\cite[p. 5]{abowd1999towards} are categorized as the \emph{primary} context types that ``answer  the questions of who, what, when, and where''~\cite[p. 5]{abowd1999towards}. Answering these questions help us understand \emph{why} a given situation occurs. In this project an action is triggered on a controllable device because the user is interested in triggering it (that is the \emph{why}).

\emph{Secondary} context types can be derived from the \emph{primary} context types. When the identity of a user is known, additional information can be derived. In the previous example the sex and age of the user are primary context types that can be derived from the identity of the user. 
The price reduction of the t-shirt is secondary information as well as it can be derived from the identity of the t-shirt.
The position of the user and the t-shirt are both primary context types.

\subsection{Context Features}

Abowd \etal\cite{abowd1999towards} defines a context-aware system as in \Cref{def:context-aware}.

\begin{definition}
\label{def:context-aware}
``A system is context-aware if it uses context to provide relevant information and/or services to the user, where relevancy depends on the user's task.''~\cite[p. 6]{abowd1999towards}
\end{definition}

Based on this definition,~\cite{ferreira2014distributed} outlines the following three main features a context-aware system can provide to its users.

\begin{description}
\item[Presentation of information and services] Systems with this feature use context to suggest services to the users or present them with relevant information. Yelp, a service that presents users with nearby businesses, is an example of a system implementing this feature.
\item[Automatic execution of a service] Systems with this feature automatically execute a service based on context. Philips Hue, which can automatically change the lighting based on the time of day, is an example of a system implementing this feature.
\item[Tagging of context to information for later retrieval] Systems with this feature associate information with context.~\cite{ferreira2014distributed} uses a service that tags locations with a virtual note for other users to see as an example of systems implementing this feature.
\end{description}

According to the categorization of context-aware systems by Ferreira \etal~the system envisioned in this report belongs to the category of systems implementing automatic execution of a service. Based on context, the system automatically triggers an action on a controllable device. While the user must perform a gesture in order to trigger the action, the system is still automatic as we consider the gesture to be context.

\subsection{Conclusion}

We follow the definitions of context and context-aware systems proposed by Abowd \etal. We also introduce the term \emph{contextual information} as defined in \Cref{def:contextual-information}. 

\begin{definition}
\label{def:contextual-information}
Contextual information is information that is part of the context, \eg~the position of the user.
\end{definition}

Examples of contextual information, include the following.

\begin{itemize}
\item The day of the week.
\item The time of the day.
\item The position of the user.
\item The mood of the user.
\item The sex of the user.
\item The age of the user.
\item Users present, \eg~users that are home.
\item Sentences or words pronounced by the user.
\item Motion gestures made by the user.
\item The state of the controllable devices in the system.
\end{itemize}

While other context can be included in the system, for the purpose of the prototype presented in this report, we chose to focus on the following two contextual informations.

\begin{itemize}
\item The position of the user.
\item The gesture performed by the user.
\end{itemize}

The contextual information provides us with a way of initiating the process of determining the correct action to trigger. When the user has performed a motion, we begin the recognition of the context. Gestures can trigger different actions depending on the position of the user, thus reducing the number of gestures the user needs to remember as a single gesture can be configured to trigger different actions as illustrated in \Cref{sec:analysis:scenarios}.

We refer to the combination of gesture, room and as a \emph{gesture configuration} as defined in \Cref{def:gesture-configuration}. Gesture configurations are meant to be configured by a user. For example, a user can configure a circle gesture to turn on table lamp when he is in the living room.

\begin{definition}
\label{def:gesture-configuration}
A \emph{gesture configuration} is the combination of a gesture, a room and an action. When the user performs the gesture in the specified room, the selected action is triggered in the system.
\end{definition}

The system is context-aware as it automatically executes a service when a gesture is performed, provided that an appropriate action to trigger can be determined.

%%% Local Variables:
%%% mode: latex
%%% TeX-master: "../../master"
%%% End:
