\section{Choice of Gesture Recognizer}
\label{sec:analysis:choice-of-gesture-recognizer}
\todo[author=Kasper]{Jeg var ikke helt sikker på hvor det her afsnit skulle være, så vi må lige overveje om det skal være et andet sted, måske design.}

In \cite{prespecialisation} we stated that are two primary methods of gesture recognition, camera-based and motion-based.
\cref{sec:related-work} as well as \cite{prespecialisation} presented different implementations of camera-based solutions, however as stated in both we opted to not use this method as it requires a sightline between the user and any device he intends to interact with.

Hence we elected to use a motion-based approach.
In \cite{prespecialisation} we argued that a significant amount (55\%) of wrist-worn wearables, according to Vandrico\footnote{Vandricos data can be found at \url{http://vandrico.com/wearables/}}, contained an accelerometer and as such it makes sense to focus on motion-based gesture recognition utilizing that.
\todo[author=Kasper]{Skal vi evt. forklare hvorfor vi kun kiggede på wrist-worn? Jeg ved godt at vi i \cref{sec:analysis:choice-of-wearables} argumenterer for at det er mere naturligt at bruge sine hænder, men hvorfor kigger vi så ikke på wearables der sidder på \emph{hænderne} eller fingrene, fremfor på håndleddet?}

In \cite{prespecialisation} we used the \$3 recognizer \cite{threedollar} which is based on the \$1 recognizer \cite{wobbrock2007gestures}.
Both are designed to be simple and easy to implement and the main difference between them is that \$1 is designed for two-dimensional gestures drawn on a screen whereas \$3 is designed for three-dimensional gestures captured using a tri-axis accelerometer.

Though \$3 worked adequately for our project, we decided to search for a possibly better alternative.
We have the following criteria:

\begin{itemize}
    \item It must utilize an accelerometer to detect gesture motion data
    \item It must run on Android Wear
    \item It must recognize a gesture faster than 200ms
    \item It must support userdefined gestures
    \item Preferably it should run on the wearable independently from the smartphone
\end{itemize}
Based on these criteria we will examine the following gesture recognition solutions:

\begin{itemize}
    \item GRT \cite{gillian2011gesture, gillian2014gesture, gilliangesturegithub}
    \item \$3 \cite{threedollar}
    \item 1\textcent \cite{herold20121}
\end{itemize}
\todo[author=Kasper]{Ved ikke om jeg bør smide Kiwi og FocusMotion på listen eller om jeg blot skal smide dem ind et sted som ``related, men ikke kandidater''}