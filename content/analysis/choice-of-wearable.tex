\section{Choice of Wearable}
\label{sec:analysis:choice-of-wearables}

The wearable is primarily used for gesture recognition and positioning of the user. Furthermore the wearable is utilised to give feedback to the user when a gesture could not be recognized and for creation of gesture configurations. The wearable should allow for interactions, such as virtually changing the position of the user as described in \Cref{sec:analysis:scenarios}.

Hence we decided on the following requirements.

\begin{itemize}
\item The wearable must possess a screen on which visual feedback can be presented and allow for presenting and interacting with proposed actions, in case we are unable to determine one specific action to be triggered. Furthermore, displaying and interacting with information is needed when the user virtually changes his position.
\item The wearable must possess an accelerometer in order to detect the users movements and derive gestures from those.
\item As described in \Cref{sec:analysis:indoor-positioning}, Bluetooth is used for positioning the user. Therefore the wearable should possess Bluetooth in order to position the user.
\item WiFi connection in order to send commands to the hub. The wearable should ideally be independent of the phone and should send commands directly to the hub. WiFi is suitable for this as it is assumed that a WiFi connection is available in the entire house of the user. Had we used Bluetooth for communication between the wearable and the hub, it is likely there would often be connection issues as Bluetooth signals are significantly weakened by walls and furniture.
\item We must be able to install and run our own applications on the wearable.
\end{itemize}

As we want a wearable that allows for both performing gestures, providing visual feedback to the user and let the user interact with an application running on the wearable, we chose to focus on smartwatches.

Performing gestures using a hand feels more natural than the feet, head or other body parts. This makes smartwatches ideal for recognizing gestures as they are worn on the wrist. Furthermore smartwatches are typically equipped with a touchscreen or buttons to interact with a screen or both.

The following wearables were chosen based on their popularity, availability and a desire to include a watch by each of the major players in the mobile market which, as of writing is Google, Microsoft, Apple and Samsung. Furthermore we included a Pebble Classic in our analysis as we already had access to one.

\begin{itemize}
\item Pebble Classic running Pebble OS.
\item Second generation Moto 360 running Android Wear.
\item Samsung Gear S2 running Tizen.
\item Apple Watch running watchOS.
\item Microsoft Band running Microsofts wearable architecture.
\end{itemize}

\Cref{tbl:analysis:choice-of-wearable} shows a comparison of the smartwatches based on the parameters previously listed. The table shows that the Pebble Classic and the Apple Watch do not provide access to the Bluetooth API and thus we cannot perform positioning directly on the smartwatch. An alternative is to accept this limitation and perform the positioning of the user on a smartphone and continously transfer the positions to the hub. The smartwatch can then retrieve the positions when needed.
This limitation requires the user to carry the phone wherever he goes.

\renewcommand{\arraystretch}{1.2}
\begin{table}[htb]
\centering
\caption{Comparison of wearables}
\label{tbl:analysis:choice-of-wearable}
\begin{tabularx}{\linewidth}{rXX}
\multicolumn{1}{l}{\textbf{}}     & \textbf{Pebble Classic}                       & \textbf{Second generation Moto 360} \\
\textbf{Feedback} & Screen with hardware buttons for interaction. & Touchscreen.                        \\
\textbf{Gesture recognition}      & Accelerometer.~\cite{pebble:accelerometer}                                & Accelerometer and gyroscope available from API level 3 and 9 respectively.~\cite{motorola:moto360, android:creating-wearable-apps, android:motion-sensors, android:sensors-overview}        \\
\textbf{Positioning}              & No access to the Bluetooth API.               & Access to the Bluetooth API available from API level 1.~\cite{motorola:moto360, android:creating-wearable-apps, android:Bluetooth, android:broadcast-receiver}        \\
\textbf{WiFi}                     & Unknown\footnotemark[\getrefnumber{ftn:choice-of-wearable:unknown-feature}].    & Yes.                                \\
\multicolumn{1}{l}{}              & \textbf{Samsung Gear S2}                      & \textbf{Apple Watch}                \\
\textbf{Feedback} & Touchscreen with a rotating bezel.             & Touchscreen with digital crown.     \\
\textbf{Gesture recognition}      & Accelerometer and gyroscope.~\cite{samsung:gears2, tizen:sensors}                  & Accelerometer.                      \\
\textbf{Positioning}              & Access to the Bluetooth API.~\cite{samsung:gears2, tizen:Bluetooth}                  & No access to the Bluetooth API.     \\
\textbf{WiFi}                     & Yes.                                          & Yes.                               \\
\multicolumn{1}{l}{}              & \textbf{Microsoft Band}                      & ~                \\
\textbf{Feedback} & Touchscreen.            & ~     \\
\textbf{Gesture recognition}      & Accelerometer and gyroscope.~\cite{microsoft:band-sdk, microsoft:band-sdk-documentation}                  & ~                      \\
\textbf{Positioning}              & Unknown\footnotemark[\getrefnumber{ftn:choice-of-wearable:unknown-feature}].                  & ~     \\
\textbf{WiFi}                     & No.                                          & ~                               
\end{tabularx}
\end{table}

\footnotetext{\label{ftn:choice-of-wearable:unknown-feature}We found no reference to this in the official documentation and therefore assume that the feature is not available.}

\subsection{Conclusion}

\Cref{tbl:analysis:choice-of-wearable} shows that only two wearables fulfill our requirements, namely the second generation Moto 360 by Motorola and the Gear S2 by Samsung. The other wearables provide access to the an accelerometer but do not provide access to the Bluetooth API. 

Of the two proposed wearables, we find the Moto 360 to be best suited for our project as we already have experience with the Android platform and therefore should be able to make progress on a prototype of the system faster.

%%% Local Variables:
%%% mode: latex
%%% TeX-master: "../../master"
%%% End:
