\section{Context Engine}
\label{sec:analysis:context-engine}
\todo[author=Kasper]{Overvej at omdøbe Context Engine til noget mere passende}

As presented in \cref{sec:requirements-specification}, the system should be context-aware in order to ensure that the correct actions are triggered on the correct devices in accordance with the users intention.
To accomplish this we propose what we call a \emph{context engine}.
This context engine should ideally encompass all of the contextual information regarding the system, such as which gesture the user performed and where he is located, but as mentioned in \cref{sec:analysis:context} we only use a subset of information. 
Using this information the context engine will determine which device the user intended to control as well as which action he intended to perform on it.
Since gestures are part of the context, the user can use the same gesture for multiple devices assuming the entire context is not identical.
This allows for a reduced set of gestures that the user has to recall.

Examples of other information that could be considered part of the context include:

\begin{itemize}
\item The state of the controllable devices in the system.
\item The day of the week.
\item The time of the day.
\end{itemize}

We suggest a design of the context engine based on the following requirements.

\begin{itemize}
\item While this project is focused on combining gesture recognition with the position of a user, the design should make no assumptions about the supplied type of information about the context.
\item The design should support suggesting multiple actions if it is unsure which action the user desired to trigger.
\item The design should support an arbitrary number of informations about the context with zero being an exception as it does not make sense to determine the context with no information about it.
\end{itemize}

The context engine can be regarded as a recommender system which given some input, \eg~the gesture performed by the user and his position in the smart home, recommends one or more actions the user may desire to trigger. \cref{sec:analysis:recommender-methods:collaborative-filtering} describes the technique chosen for the context engine. \cref{sec:design:bayesian-network} describes the design of the context engine.

%%% Local Variables:
%%% mode: latex
%%% TeX-master: "../../master"
%%% End:
