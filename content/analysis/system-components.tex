\section{System Components}
\label{sec:analysis:system-components}

The following section summarize the hardware components needed in order to design and implement the scenarios presented in \Cref{sec:analysis:scenarios}.

We envision two different methods for positioning of the system. Both metods assume that bluetooth technology is utilized to determine the position of the user as described in \cref{sec:analysis:indoor-positioning}.

\begin{enumerate}
\item One configuration in which the wearable continously scan for bluetooth beacons. The wearable determines the position of the user based on data advertised by the bacons.
\item Another configuration in which bluetooth enabled microcontrollers scan for wearables and uploads the RSSI to a central location in which the position of the user is determined based on the set of available RSSIs.
\end{enumerate}

The idea of the second configuration is to design the system in such a way that it works on wearables that do not provide access to the bluetooth APIs. A microcontroller should do one of the following.

\begin{itemize}
\item It should either continously scan for wearables and read their RSSI.
\item Given the MAC address of the wearable, it should obtain the RSSI.
\end{itemize}

The second configuration was abandoned due to restrictions posed by the Bluetooth LE specification. The specification poses the following limitations that prevent such a system from working properly.

\begin{itemize}
\item A Bluetooth LE peripheral (\eg~a wearable) can only be the slave of one master, that is, it can only be paired with a single other device.
\item A peripheral must advertise in order to be discovered by a central. Advertising requires a piece of software to be running on the peripheral and thus access to the bluetooth API.
\item A central can obtain an RSSI based on the MAC address of a peripheral. However, the bluetooth specification allows a peripheral to change its MAC address in order to prevent tracking of the user\cite[p.~91]{bluetooth2010bluetooth_vol_1}. The change of MAC address could be disabled for this approach, if this is possible on the wearable. However, since this is a security feature introduced in the bluetooth specification, this seems counterintuitive.
\end{itemize}

\todo[author=Simon]{Add references on above claims.}

The rest of the report will focus on the first configuration of the system.

\subsection{Required Hardware}
\label{sec:analysis:system-components:required-hardware}

The following list presents the hardware needed for the system.

\begin{itemize}
\item A computer running the hub. The hub is responsible for forwarding commands from the user to the controllable devices, \eg~lamps. The benefit of running the hub on a central computer rather than the phone, is that the hub could potentially be configured with rules for automation and should therefore always be running, compromising the battery life of a wearable. Furthermore placing the logic in a central place can prove benefitial in an environment with multiple users in which multiple hubs would have to be synchronized.
\item A wearable capable which provides access to APIs for both bluetooth and the accelerometer. Furthermore it should be possible to give some sort of feedback to the user when a gesture could not be recognized. The wearable should also be able to communicate with hub.
\item Minimum one bluetooth beacon per room and minimum two beacons in order to configure the system with two rooms. The bluetooth beacons are used to determine which room the user is in.
\item Minimum two controllable devices, one per room in the system. These devices receives requests from the hub that ask them to change their state, \eg~turn on or off in case of a lamp.
\end{itemize}

The above lists the bare minimum of hardware required in order to implement the system. \Cref{sec:analysis:choice-of-wearables,sec:analysis:choice-of-hub,sec:analysis:indoor-positioning} elaborates on the choice of hardware components.

%%% Local Variables:
%%% mode: latex
%%% TeX-master: "../../master"
%%% End:
