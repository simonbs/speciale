\section{Hardware Components}
\label{sec:analysis:system-components}

The following section summarize the hardware components needed in order to design and implement the scenarios presented in \Cref{sec:analysis:scenarios}.

We envision two different configurations for positioning of the system. Both configurations assume that Bluetooth technology is utilized to determine the position of the user as described in \Cref{sec:analysis:indoor-positioning}.

\begin{enumerate}
\item One configuration in which the wearable continuously scan for Bluetooth beacons. The wearable determines the position of the user based on data advertised by the bacons.
\item Another configuration in which Bluetooth enabled microcontrollers scan for wearables and uploads the RSSI to a central location in which the position of the user is determined based on the set of available RSSIs.
\end{enumerate}

The idea of the second configuration is to design the system in such a way that it works on wearables that do not provide access to the Bluetooth APIs. A microcontroller should do one of the following.

\begin{itemize}
\item It should either continously scan for wearables and read their RSSI.
\item Given the MAC address of the wearable, it should obtain the RSSI.
\end{itemize}

The second configuration was abandoned due to restrictions posed by the Bluetooth Low Energy (BLE) specification. The specification poses the following limitations that prevent such a system from working properly.

\begin{itemize}
\item A BLE peripheral, \eg~a wearable, can only be paired with a single other device.
\item A peripheral must advertise in order to be discovered by a central. Advertising requires a piece of software to be running on the peripheral and thus access to the Bluetooth API.
\item A central can obtain an RSSI based on the MAC address of a peripheral. However, the Bluetooth specification allows a peripheral to change its MAC address in order to prevent tracking of the user~\cite[p.~91]{Bluetooth2010Bluetooth_vol_1}. The change of MAC address could be disabled for this approach, if this is possible on the wearable. However, since this is a security feature introduced in the Bluetooth specification, this seems undesirable for our purposes.
\end{itemize}

The rest of the report will focus on the first configuration of the system.

\subsection{Required Hardware}
\label{sec:analysis:system-components:required-hardware}

The following list presents the hardware needed for the system.

\begin{itemize}
\item A computer running the hub. The hub is responsible for forwarding commands from the user to the controllable devices, \eg~lamps. The benefit of running the hub on a central computer rather than the smartwatch, is that the hub could potentially be configured with rules for automation and should therefore always be running, compromising the battery life of a wearable. Furthermore placing the logic in a central place can prove beneficial in an environment with multiple users in which multiple hubs would have to be synchronized.
\item A wearable which provides access to APIs for both Bluetooth and the accelerometer. Furthermore it should be possible to give some sort of feedback to the user when a gesture could not be recognized. The wearable should also be able to communicate with hub.
\item Minimum one Bluetooth beacon per room. Two beacons are needed in other to test the system in more than one room. The Bluetooth beacons are used to determine which room the user is in.
\item Minimum two controllable devices, one per room in the system. These devices receive requests from the hub that ask them to change their state.
\end{itemize}

The above lists the bare minimum of hardware required in order to implement the system. \Cref{sec:analysis:choice-of-wearables,sec:analysis:choice-of-hub,sec:analysis:indoor-positioning} elaborates on the choice of hardware components.

\section{Accelerometer}

Accelerometers are used for measuring the acceleration of an object. In our case we use it to measure the acceleration of the wearable installed on a users hand and as a result of this, the users movements. When measuring the users movements, we can recognize gestures he performs.

When an object is subjected to a force, including gravity, it accelerates. Acceleration is a vector indicating a change in velocity. The acceleration can be expressed as

\begin{equation*}
a = \frac{F}{m}
\end{equation*}

where $a$ is the acceleration of the object. $F$ is the force applied to the object expresses as a vector with a force for each axis. The forces are measured in Newton. $m$ is the mass expressed as a scalar value.

In order to calculate the acceleration, the mass of the object must be known. The accelerometer determines the force applied to the object \cite[pp. 392-393]{Fraden:2112745}.

Accelerometers of varying design exist \cite[pp. 392-411]{Fraden:2112745} an example of these is the capacitive type semiconductor accelerometer which is illustrated in figure \ref{fig:accelerometer}. The accelerometer has an electrode in the middle (14) which is supported by a beam (13). The beam is flexible enough that it moves slightly up and down when the accelerometer is moved, thus moving the electrode up and down. The beam is mounted to the side of the accelerometer (2). The gap between the electrode (14) and the two stationary electrodes (25, 26) constitute two electric capacitors having capacitances of $C_1$ and $C_2$.
When the movable electrode in the middle (14) moves up and down, the capacitances $C_1$ and $C_2$ changes slightly \cite{kloeck1993capacitive}.

The changes are registered and constitutes the acceleration. This allows for measuring the acceleration on one axis. The same principle can be used to measure the acceleration on several axes.

\begin{figure}
\centering
\includegraphics[width=0.5\textwidth]{images/accelerometer}
\caption{Capacitive type semiconductor accelerometer \cite{kloeck1993capacitive}.}
\label{fig:accelerometer}
\end{figure}


%%% Local Variables:
%%% mode: latex
%%% TeX-master: "../../master"
%%% End:

\subsection{Bluetooth}
\label{sec:analysis:bluetooth}

We use Bluetooth Low Energy (BLE) in order to position the user wearing the smartwatch. Bluetooth is a standard for short-range and low-power wireless communication between devices and is commonly found in a broad range of devices including desktop computers, phones and speakers~\cite{gupta2013inside}.

Bluetooth has a maximum range of 100 meters but is typically used for much shorter distances~\cite[p. 20]{gupta2013inside}. Accomplishing the maximum range is difficult as walls, furniture and people can dampen the strength of the signals~\cite{faragher2014analysis}.

\Cref{fig:analysis:bluetooth:highlevel-architecture} shows the high-level layered architecture of Bluetooth. The layers are described below.

\begin{description}
\item[Lower layers] Perform low level operations, including discovering devices, establishing connections and exchanging data packets. The functionality is implemented on the Bluetooth chip~\cite[pp. 21-22]{gupta2013inside}. We will not go into details about this layer.
\item[Upper layers] Use functionality of the lower layers to perform complex functionality, including transferring large amounts of data by splitting it into multiple packets and streaming of data~\cite[p. 22]{gupta2013inside}.
\item[Profiles] The profiles define how the protocol layers within the upper and lower layers implement specific use cases, \eg~proximity detection~\cite[p. 22]{gupta2013inside}.
\item[Applications] These are applications utilizing the Bluetooth stack, \eg~mechanisms for discovering and connecting to Bluetooth devices, choosing music to stream and selecting files to transfer~\cite[p. 22]{gupta2013inside}.
\end{description}

\begin{figure}[!htb]
\centering
\includegraphics[width=0.8\textwidth]{images/bluetooth-architecture}
\caption{High-level architecture of Bluetooth. Illustration from~\cite[p. 22]{gupta2013inside}.}
\label{fig:analysis:bluetooth:highlevel-architecture}
\end{figure}

As part of the upper layers, is the Logical Link Control and Adaption Protocol, abbreviated L2CAP. The protocol builds on top of protocols in the lower levels, to exchange data with a remote Bluetooth device. L2CAP provides functionality that includes segmentation and reassembly of packets, quality of service, streaming data and retransmission of packets. Devices communicating using L2CAP exchange Packet Data Unit (PDU) packets, containing information about the L2CAP protocol, \eg~the type of the PDU and a payload~\cite[pp. 80-83]{gupta2013inside}. For example, the PDU type of a beacon can be \texttt{ADV\_NONCONN\_IND}, indicating a non-connectable undirected advertising packet. The payload of a PDU is referred to as the Service Data Unit (SDU) which originates from a level above the L2CAP, \eg~the Attribute Protocol~\cite[p. 201]{gupta2013inside}.

Attribute Protocol, abbreviated ATT, uses L2CAP to transfer data. Mechanisms provided by ATT include discovering the attributes provided by a remote device and reading and writing the attributes. An attribute represents data, for example the temperature from a thermostat, the unit in which the temperature is provided or the name of a device. Attributes can be pushed or pulled to and from a remote device. Attributes have a handle that identifies an attribute, a value and access permissions. The protocol works in a client-server manner in which a server exposes a set of attributes and a client can read and write the attributes~\cite{gupta2013inside}.

The Bluetooth Core 4.0 Specification~\cite{bluetooth2010bluetooth_7_vol}, \ie~the specification including BLE, introduces the General Attribute Profile (GATT) architecture illustrated in \Cref{fig:analysis:bluetooth:gatt-architecture}. The GATT framework specifies how a device can discover, read, write and indicate its characteristics. Profiles consist of one or more services that are needed in order to provide a specific functionality, \eg~proximity monitoring~\cite[p. 259-261]{gupta2013inside}. Services provide one or more characteristics that describe a feature, \eg~the temperature of a thermostat. Services may be shared by multiple profiles.

The Bluetooth Special Interest Group, the group that maintains the Bluetooth standards, provide a range of profiles, \eg~a proximity profile for monitoring the distance to devices and a profile for heart rate sensors.
Apple and Google have developed the iBeacon and Eddystone profiles, respectively, for region monitoring.
Eddystone GATT profile defines the Eddystone Service that advertises \emph{frames} of information. These are described in detail in \Cref{sec:design:ble-positioning} as well as the Eddystone Configuration Services in which a device is connectable and the advertised values can be configured by a client BLE-enabled device.

\begin{figure}[!htb]
\centering
\includegraphics[width=0.8\textwidth]{images/gatt-architecture}
\caption{Relationship between profiles, services and characteristics. Illustration from~\cite[p. 261]{gupta2013inside}.}
\label{fig:analysis:bluetooth:gatt-architecture}
\end{figure}

%%% Local Variables:
%%% mode: latex
%%% TeX-master: "../../master"
%%% End:


%%% Local Variables:
%%% mode: latex
%%% TeX-master: "../../master"
%%% End:
