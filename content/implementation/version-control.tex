\section{Version Control}
\label{sec:implementation:version-control}

We use version control management for keeping track of changes to the codebase developed through the project as well as this report. Below is a list of advantages of using version control management.

\begin{itemize}
\item Colalborators can work on the same file at the same time, due the way changes to the file can be \emph{merged}. This is in contrary to a shared folder, e.g. a Dropbox folder which always synchronize the most recent version of a file with a central location.
\item Keeping track of what files were changed, what was changed in the files and who changed it.
\item When publishing changes to the files, authors typically tag the changes with a message with a description of the changes making their intention clear to collaborators.
\item Changes can be rolled back to a previous state.
\item Collaborators of a project can \emph{branch} out from the main codebase to create changes without touching the currently stable code base. When their changes are done, they can \emph{merge} in their changes to the stable codebase.
\item Depending on the amount of collaborators and the system used, the codebase is inherently backed up.
\end{itemize}

There are several software solutions for version control management, including Git, Subversion, CVS and Bazaar. When choosing a system to use, we decided to only look into the Git and Subversion as we have experience with the two.

The key difference between Git and Subversion is, that Git is decentralized and Subversion is centralized. When using Git, collaborators have a local copy of the entire repository in which the codebase resides. Collaborators then push their changes to a central location when they are done working on a feature or a fix. With Subversion, collaborators are working in a central online repository, meaning that the version control features are unavailable when there is no connection to the repository.

We chose Git over Subversion because of it being decentralized. This provides two advantages over Subversion.

\begin{itemize}
\item When we are working with no or an unstable internet connection, version control features are still available. Several times during the project we have worked with no internet connection.
\item We believe that a decentralized and local system provides extra safety in terms of backups. All collaborators always have a backup of the repository on their local machines. In addition, we push the changes to GitHub, a website for hosting Git repositories, from which we also pull the changes other collaborators make.
\end{itemize}

%%% Local Variables:
%%% mode: latex
%%% TeX-master: "../../master"
%%% End:
