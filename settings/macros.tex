\makeatletter
\DeclareRobustCommand\onedot{\futurelet\@let@token\@onedot}
\def\@onedot{\ifx\@let@token.\else.\null\fi\xspace}
% Commands for ie, eg, etc... 
\def\eg{\emph{e.g}\onedot} \def\Eg{\emph{E.g}\onedot}
\def\ie{\emph{i.e}\onedot} \def\Ie{\emph{I.e}\onedot}
\def\cf{\emph{c.f}\onedot} \def\Cf{\emph{C.f}\onedot}
\def\etc{\emph{etc}\onedot} \def\vs{\emph{vs}\onedot}
\def\wrt{w.r.t\onedot} \def\dof{d.o.f\onedot}
\def\etal{\emph{et al}\onedot}

\def\figcaption{%
  \refstepcounter{figure}%
  \@dblarg{\@caption{figure}}}
\makeatother

\newcommand{\method}[2]{\sloppy\ensuremath{\texttt{{#1}(#2)}}}
\newcommand{\var}[1]{\sloppy\ensuremath{\mathit{#1}}}

\newcommand{\perc}[1]{\SI{#1}{\percent}}

\newcommand{\productname}{Point and Control with Gestures in a Smart Home}
\newcommand{\hour}[1]{\formattime{#1}{0}{0}}
\newcommand{\bigo}[1]{\ensuremath{\mathcal{O}\left(#1\right)}}
\newcommand{\threedollar}{\$3 Gesture Recognizer\xspace}

\newcommand{\yes}{\ding{51}~}
\newcommand{\no}{\ding{55}~}

\newcommand{\ass}{\coloneqq} % :=, vertially centered. Requires mathtools

\newcommand{\heatmap}[1]{data/estimote-test-results/heatmaps/pdf/#1}
\newcommand{\meanerror}[2]{
  \begin{tikzpicture}
  \begin{axis}[
  height=7cm,
  width=\textwidth,
  ylabel = Distance in meters,
  xlabel = Measurements over time,
  enlargelimits=false,
  ymax=#2,
  ymin=0,
  grid=major,
  max space between ticks=20pt,
  xtick = {0, 50, 100, 150, 200, 250, 300}
  ]
  \addplot table [x expr=\coordindex, y index=0] {data/estimote-test-results/mean-error/#1};
  \end{axis}
  \end{tikzpicture}
}

\newenvironment{shadedi}{% For use with aligning the def environment
  \def\FrameCommand{\fboxsep=\FrameSep \colorbox{gray!20}}%
  \MakeFramed {\advance\hsize-1\width\FrameRestore}}%
 {\endMakeFramed}

\newtheorem{tempDef}{Definition}[chapter]
\newenvironment{definition}
  {\begin{shadedi}\begin{tempDef}}
  {\end{tempDef}\end{shadedi}}

  
\Crefname{lstlisting}{Listing}{Listings}

\renewcommand*\contentsname{Table of Contents}

%%% Local Variables:
%%% mode: latex
%%% TeX-master: "../master"
%%% End:
