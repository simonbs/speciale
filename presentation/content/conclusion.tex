\section{Conclusion}

\begin{frame}{Conclusion}{Problem Statement}

\begin{framed}
\noindent How can we design and implement a system that utilizes contextual information for controlling a smart home using a wearable in a gesture driven solution?
\end{framed}

\end{frame}

\begin{frame}{Conclusion}{Our Solution}

\begin{itemize}
	\item Combination of 1\textcent~and \$3 to recognize gestures on the wearable
	\item BLE beacons to position user
	\item Raspberry Pi and openHAB to connect wearable and controllable devices
	\item Bayesian network for determining appropriate action
\end{itemize}

\end{frame}

\begin{frame}{Conclusion}{Issues}

\begin{itemize}
	\item Model was flawed and affected configurations where a gesture was bound to multiple actions
\end{itemize}

\end{frame}

\begin{frame}{Conclusion}{Influence Diagram}
\begin{figure}[h]
\centering
\begin{tikzpicture}[scale=0.7]
\begin{axis}[
    xbar,
    bar width=7pt,
    yticklabels from table={data/OldVsInfluence.csv}{Action},
    ytick=data,
    y tick label style={
    % rotate=45,
    anchor=east,
    },
    xlabel = Success Rate (\%),
    % width=0.45\textwidth,
    xmin = 0,
    xmax = 100,
    legend style={
    at={(0,0)},
    anchor=south, at={(axis description cs:0.5,1)}}
    ]
    \addplot table[y=Row, x=Previous Model] {data/OldVsInfluence.csv};
    \addplot table[y=Row, x=Influence Diagram] {data/OldVsInfluence.csv};
    \legend{Previous Model, Influence Diagram}
\end{axis}
\end{tikzpicture}
\caption{Success rates for participant 2 in the previous model compared to the proposed influence diagram.}
\end{figure}

\end{frame}

\begin{frame}
\centering\huge\textbf{\structure{Demo}}
\end{frame}
